\chapter{ANEXO D: MANUAL DE USO DEL ASISTENTE}

Este documento constituye la guía oficial de operación para el usuario final del "Asistente de Preparación PMP". Su objetivo es maximizar el aprovechamiento de las capacidades de Inteligencia Artificial del sistema, proporcionando instrucciones detalladas, estrategias de estudio y soluciones a problemas comunes.

\section{D.1. Introducción y Alcance}

El "Asistente de Preparación PMP" no es un simple banco de preguntas, sino un \textbf{Sistema Tutor Inteligente (ITS)}. A diferencia de los métodos tradicionales pasivos, este sistema requiere una interacción activa. Este manual está diseñado para ayudarle a transitar de un estudio estático a un diálogo constructivo con la IA.

\textbf{Audiencia Objetivo:}
\begin{itemize}
\item Aspirantes a la certificación PMP (Project Management Professional).
\item Gerentes de proyecto que deseen refrescar conocimientos sobre PMBOK 7ma Edición.
\item Estudiantes de posgrado en gestión de proyectos.
\end{itemize}
\section{D.2. Primeros Pasos y Configuración}

\subsection{D.2.1. Requisitos del Sistema}
Para garantizar una experiencia fluida, asegúrese de cumplir con lo siguiente:
\begin{itemize}
\item \textbf{Navegador Moderno:} Google Chrome (v90+), Microsoft Edge, Firefox o Safari. El sistema utiliza tecnologías web avanzadas (Streams, WebSockets) que no funcionan en navegadores obsoletos como Internet Explorer.
\item \textbf{Dispositivo:}
\item \textit{Escritorio/Laptop:} Recomendado para sesiones de simulación de examen y talleres de documentos.
\item \textit{Móvil/Tablet:} Ideal para sesiones de repaso rápido (micro-learning) en transporte público o tiempos muertos.
\item \textbf{Conexión:} Internet estable de al menos 5 Mbps.
\end{itemize}
\subsection{D.2.2. Proceso de Onboarding (Inicio)}
Al acceder por primera vez, el sistema lanzará automáticamente el asistente de configuración:
\begin{enumerate}
\item \textbf{Identificación:} Ingrese el nombre con el que desea ser tratado. La IA utilizará este nombre para personalizar los ejemplos (ej. "Juan, imagina que eres el gerente de...").
\item \textbf{Calibración:} El sistema le presentará brevemente la estructura de "Mundos" (Dominios de Desempeño).
\item \textbf{Confirmación:} Al finalizar, se creará su perfil local y se le otorgarán sus primeros 100 XP (Puntos de Experiencia) por completar el registro.
\end{enumerate}
\section{D.3. Navegación por la Interfaz (Dashboard)}

El Dashboard ha sido diseñado bajo el principio de "Divulgación Progresiva": solo muestra lo que necesita ver en su nivel actual, evitando la sobrecarga cognitiva.

\subsection{D.3.1. Barra Lateral de Progreso (Sidebar)}
Ubicada a la izquierda (o en el menú hamburguesa en móviles), representa su hoja de ruta.
\begin{itemize}
\item \textbf{Fases:} El contenido se agrupa en 4 fases lógicas (Principios, Personas, Procesos, Entorno).
\item \textbf{Indicadores de Estado:}
\item \checkmark \textbf{Verde:} Nivel completado y aprobado.
\item 🔵 \textbf{Azul/Resaltado:} Nivel actual en curso.
\item \faLock \textbf{Gris/Candado:} Nivel bloqueado. Requiere aprobar el anterior.
\end{itemize}
\subsection{D.3.2. Modos de Operación (Selector de Vistas)}
El Dashboard cuenta con un selector en la parte superior que permite cambiar la visualización y funcionalidad del área principal según su objetivo de estudio:

\begin{enumerate}
\item \textbf{🗺️ Modo Guiado:}
\end{enumerate}
\begin{itemize}
\item \textbf{Descripción:} Es el modo por defecto y recomendado para el aprendizaje inicial.
\item \textbf{Funcionamiento:} Presenta el contenido de forma lineal. Bloquea fases y niveles futuros hasta que se completen los requisitos previos.
\item \textbf{Ventaja:} Reduce la ansiedad y asegura una progresión pedagógica lógica.
\end{itemize}
\begin{enumerate}
\item \textbf{🔓 Modo Desbloqueado:}
\end{enumerate}
\begin{itemize}
\item \textbf{Descripción:} Mantiene la estructura visual del Modo Guiado (Fases y Mundos) pero elimina todas las restricciones.
\item \textbf{Funcionamiento:} Todas las fases y niveles están accesibles inmediatamente.
\item \textbf{Uso Ideal:} Para usuarios que regresan a repasar temas específicos o expertos que no necesitan la secuencia obligatoria.
\end{itemize}
\begin{enumerate}
\item \textbf{♾️ Modo Libre:}
\end{enumerate}
\begin{itemize}
\item \textbf{Descripción:} Elimina la estructura de "Mundos" para ofrecer un acceso directo a las herramientas de IA ("Entrenamiento Estándar").
\item \textbf{Catálogo de Herramientas (Grid Principal):}
\item \textbf{\faBook Modo Estándar:} El asistente clásico. Preguntas y respuestas directas sobre cualquier tema del PMBOK. \textbf{Cuándo usarlo:} Para obtener definiciones claras, diferencias entre conceptos o resúmenes. \textbf{Ejemplo:} \textit{"¿Cuál es la diferencia principal entre el Acta de Constitución y el Enunciado del Alcance?"}
\item \textbf{\faExclamationTriangle Simulación de Crisis:} Roleplay inmersivo donde la IA actúa como un stakeholder difícil o un equipo en problemas. Usted asume el rol de PM para resolver la situación en tiempo real.
\item \textbf{\faWrench Taller de Entregables:} Herramienta de creación guiada. Redacte Project Charters, matrices de riesgo y planes de gestión paso a paso con la ayuda experta del asistente.
\item \textbf{\faPencilSquareO Examen Rápido:} Póngase a prueba con preguntas tipo PMP aisladas. Reciba feedback inmediato y explicaciones detalladas de cada respuesta (correcta o incorrecta).
\item \textbf{\faBrain Tutor Socrático:} Para profundizar en conceptos. La IA no le dará la respuesta directa, sino que le guiará con preguntas reflexivas ("Mayéutica"). \textbf{Cuándo usarlo:} Cuando memorizar no es suficiente. \textbf{Comportamiento:} \textit{"Piénsalo en términos de dinero vs. tiempo. ¿Qué puedes recuperar...?"}
\item \textbf{\faGavel Debate (Abogado del Diablo):} Defienda sus ideas. La IA tomará intencionalmente una postura polémica. \textbf{Ejemplo:} Si usted afirma que \textit{"Ágil es siempre mejor"}, la IA le preguntará: \textit{"¿Qué harías en un proyecto nuclear con requisitos fijos? Defiende tu punto."}
\item \textbf{\faBriefcase Caso de Estudio:} Análisis de escenarios complejos. \textbf{Funcionamiento:} La IA genera un problema breve (5-10 líneas, ej. conflicto de stakeholders) y usted debe actuar como consultor externo para proponer una solución basada en PMBOK.
\item \textbf{\faChild Explícamelo como a un niño (ELI5):} Utiliza analogías cotidianas (ej. legos, cocina) para explicar temas densos. \textbf{Ejemplo (EVM):} \textit{"Imagina que construyes una casa de Lego. Tenías que poner 10 ladrillos hoy (Valor Planificado), pero solo pusiste 8..."}
\item \textbf{\faCalculator Entrenador de Fórmulas:} Módulo especializado para dominar el Valor Ganado (EVM) y la Ruta Crítica. Practique ejercicios numéricos y aprenda a interpretar los resultados (CPI, SPI, TCPI).
\item \textbf{Uso Ideal:} Para sesiones de práctica enfocada (ej. "Hoy solo quiero practicar fórmulas matemáticas").
\end{itemize}
\begin{enumerate}
\item \textbf{\faGraduationCap Simulación Examen:}
\end{enumerate}
\begin{itemize}
\item \textbf{Descripción:} Entorno dedicado exclusivamente a la evaluación.
\item \textbf{Funcionalidades:}
\item \textbf{Panel de Métricas:} Visualización gráfica del rendimiento acumulado y por dominios.
\item \textbf{Historial:} Acceso a simulaciones pasadas para revisión o continuación.
\item \textbf{Lanzador:} Inicio de simulacros cronometrados (Parciales o Completos).
\end{itemize}
\section{D.4. Simulador de Examen y Métricas}

El módulo de simulación es su herramienta de validación final.

\subsection{D.4.1. Tipos de Examen}
\begin{enumerate}
\item \textbf{Examen de Nivel (Micro-evaluación):}
\end{enumerate}
\begin{itemize}
\item \textit{Duración:} ~5 minutos.
\item \textit{Contenido:} 3 preguntas enfocadas exclusivamente en el tema del nivel actual.
\item \textit{Objetivo:} Desbloquear el siguiente nivel.
\end{itemize}
\begin{enumerate}
\item \textbf{Simulacro Completo (Exam Simulator):}
\end{enumerate}
\begin{itemize}
\item \textit{Opciones:} 45, 90, 135 o 180 preguntas.
\item \textit{Cronómetro:} Activo (promedio 1.2 minutos por pregunta).
\item \textit{Feedback:} Solo disponible al finalizar el examen completo.
\end{itemize}
\subsection{D.4.2. Interpretación de Resultados}
Al finalizar, verá un desglose por Dominio de Desempeño:
\begin{itemize}
\item \textbf{Dominio Personas:} Liderazgo, conflicto, equipo.
\item \textbf{Dominio Procesos:} Gestión técnica, riesgos, costos, cronograma.
\item \textbf{Dominio Entorno de Negocio:} Cumplimiento, valor, cambio organizacional.
\end{itemize}
\textbf{Recomendación:} Si su puntaje en un dominio es <60%, el sistema le sugerirá volver a los niveles correspondientes a ese dominio antes de reintentar.

\section{D.5. Rutinas de Estudio Recomendadas}

Para maximizar la eficiencia, sugerimos los siguientes flujos de trabajo según su disponibilidad:

\subsection{Rutina A: "El Sprint Diario" (30 min)}
\begin{enumerate}
\item \textbf{5 min:} Revisión de conceptos del día anterior con el \textbf{Tutor Socrático}.
\item \textbf{15 min:} Avance de un nuevo sub-tema en el \textbf{Nivel Actual}.
\item \textbf{10 min:} \textbf{Examen de Nivel} o repaso rápido con \textbf{ELI5} si el tema fue difícil.
\end{enumerate}
\subsection{Rutina B: "Inmersión de Fin de Semana" (2-3 horas)}
\begin{enumerate}
\item \textbf{45 min:} Sesión profunda de \textbf{Caso de Estudio} o \textbf{Taller de Entregables}.
\item \textbf{60 min:} \textbf{Simulacro Parcial} (45 o 90 preguntas).
\item \textbf{30 min:} Análisis de fallos. Tome las preguntas que falló en el simulacro y discútalas con el \textbf{Modo Debate} para entender por qué su lógica fue incorrecta.
\end{enumerate}
\section{D.6. Gamificación y Motivación}

El sistema utiliza elementos de juego para mantener la motivación:
\begin{itemize}
\item \textbf{XP (Experiencia):} Se gana por cada interacción útil, examen aprobado y nivel completado.
\item \textbf{Rachas (Streaks):} Días consecutivos de estudio. Perder una racha puede afectar la bonificación de XP.
\item \textbf{Confeti:} Celebración visual al superar hitos importantes (aprobar un nivel difícil).
\item \textbf{Feedback Positivo:} La IA está programada para reforzar el esfuerzo, utilizando frases motivadoras cuando detecta frustración.
\end{itemize}
\section{D.7. Solución de Problemas y Preguntas Frecuentes (FAQ)}

\subsection{D.7.1. Problemas Técnicos}
\begin{itemize}
\item \textbf{Error: "Connection timeout" o IA no responde.}
\item \textit{Causa:} Interrupción momentánea de red o sobrecarga de la API.
\item \textit{Solución:} Refresque la página (F5). El historial de chat reciente se recuperará de la base de datos.
\item \textbf{Error: "Application Error: Client-side exception".}
\item \textit{Causa:} Posible conflicto con extensiones del navegador (AdBlockers).
\item \textit{Solución:} Intente abrir la aplicación en Modo Incógnito.
\end{itemize}
\subsection{D.7.2. Problemas de Contenido}
\begin{itemize}
\item \textbf{La IA inventó una cita o página del PMBOK.}
\item \textit{Contexto:} Los LLMs pueden tener "alucinaciones" leves con datos muy específicos.
\item \textit{Acción:} Confíe en los conceptos y lógicas, pero verifique números de página o citas textuales en su copia física del PMBOK si es crítico.
\item \textbf{El examen se cerró por accidente.}
\item \textit{Aviso:} Actualmente, los simulacros completos no tienen auto-guardado en la nube pregunta a pregunta para maximizar la velocidad. Si cierra la pestaña, deberá reiniciar el simulacro. Esta es una limitación conocida de la versión actual.
\end{itemize}
\subsection{D.7.3. Soporte}
Si encuentra un error persistente o tiene sugerencias de mejora, contacte al equipo de desarrollo a través del repositorio oficial en GitHub.
